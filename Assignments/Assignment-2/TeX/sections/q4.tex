\section*{Exercise 1.4}

Consider the pendulum DS with friction
\[
    \ddot{\theta}=-g \sin (\theta)-\dot{\theta}
\]
\begin{enumerate}[noitemsep]
    \item Write down a state space representation using variable \( x=\left(x_{1}, x_{2}\right) \).
    \item As \( x=(0,0) \) is a Lyapunov-stable equilibrium point, there exists a Laypunov candidate function \( V(x) \) such that:
          \begin{align*}
               & V(0,0)=0                                            \\
               & V(x)>0, \dot{V}(x) \leq 0 \quad \forall x \neq(0,0)
          \end{align*}
    \item Write a Lyapunov function satisfying it and show that \( (0,0) \) is stable.
    \item Compute and plot the path integral of the pendulum DS without damping \( x(0)= \) \( \binom{0}{0},\binom{0.78}{0},\binom{2.35}{0},\binom{3.14}{1},\binom{3.14}{4} \).
              [ state space range between -10 to 10]
    \item Compute and plot the path integral of the pendulum DS with damping \\
          \( x(0)= \) \( \binom{0}{0},\binom{0.78}{0},\binom{2.35}{0},\binom{3.14}{1},\binom{3.14}{4} \).
              [ state space range between -10 to 10]
\end{enumerate}

\subsection*{Solution}

\subsection*{1. State space representation}

The given system is a second-order nonlinear system which be represented as, by using the state variables \( x_{1} = \theta \) and \( x_{2} = \dot{\theta} \),
\[
    \boxed{
        \begin{bmatrix}
            \dot{x}_{1} \\
            \dot{x}_{2}
        \end{bmatrix}
        =
        \begin{bmatrix}
            x_{2} \\
            -g \sin (x_{1}) - x_{2}
        \end{bmatrix}
        ,\quad x_{1}, x_{2} \in \mathbb{R}
    }
\]

\subsection*{2, 3. Lyapunov-stable equilibrium point}

The equilibrium points can be found by
\begin{align*}
    \implies
    \dot{x}_{1}
     & =
    x_{2} = 0
    \\
    \implies
    \dot{x}_{2}
     & =
    -g \sin (x_{1}) - x_{2} = 0
    \implies
    \sin (x_{1}) = 0
    \implies
    x_{1} = n \pi, \ n \in \mathbb{Z}
\end{align*}
giving the equilibrium points as \( (n \pi, 0) \), \( n \in \mathbb{Z} \), where we are only interested in \( (0, 0) \).

The system can be linearized around the equilibrium point \( (0, 0) \) as
\[
    \begin{bmatrix}
        0             & 1  \\
        -g \cos (x_1) & -1
    \end{bmatrix} \Bigg|_{x = (0, 0)}
    =
    \begin{bmatrix}
        0  & 1  \\
        -g & -1
    \end{bmatrix}
    \implies
    \dot x = Ax, \quad
    A \triangleq \begin{bmatrix} 0 & 1 \\ -g & -1 \end{bmatrix}
\]
