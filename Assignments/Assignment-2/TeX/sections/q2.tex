\section*{Exercise 1.2}

Consider two variables \( x \) and \( y \) coupled with the following dynamics
\begin{align*}
     & \dot{x}=\beta x, \quad \beta \in R      \\
     & \dot{y}=-y+\alpha x, \quad \alpha \in R
\end{align*}
Answer the following:
\begin{enumerate}[noitemsep]
    \item Does this system have an equilibrium point? What is it?
    \item For what values of \( \alpha \) and \( \beta \) is the system stable at the equilibrium point?
    \item For what values of \( \alpha \) and \( \beta \) is the system unstable at the equilibrium point?
\end{enumerate}

\vspace*{-1em}
\subsection*{Solution}

The given system is a linear system which can be represented as
\begin{equation*}
    \begin{bmatrix}
        \dot{x} \\
        \dot{y}
    \end{bmatrix}
    =
    \begin{bmatrix}
        \beta  & 0  \\
        \alpha & -1
    \end{bmatrix}
    \begin{bmatrix}
        x \\
        y
    \end{bmatrix}
    ,\quad \alpha, \beta \in \mathbb{R}
\end{equation*}

\vspace*{-1em}
\subsubsection*{1. Equilibrium point}

The equilibrium point can be found by setting the derivatives of the state variables to zero.
\[
    \implies
    \dot{x} = \beta x = 0
    \implies
    \beta \in \mathbb{R}\setminus \{0\}, \ x = 0
    \quad \text{or} \quad
    \beta = 0, \ x = x_0, \text{ for some } x_0 \in \mathbb{R}
\]
\[
    \implies
    \dot{y} = -y + \alpha x = 0
    \implies
    y = \alpha x
    \implies
    y = \begin{cases}
        0,          & \text{if } \beta \neq 0 \\
        \alpha x_0, & \text{if } \beta = 0
    \end{cases}
\]
Therefore, the \underline{equilibrium point exists} at \( \boxed{ (0, 0) } \) for \( \beta \neq 0, \ \alpha \in \mathbb{R} \) \\
and at \( \boxed{ (x_0, \alpha x_0) } \) for \( \beta = 0, \ x_0, \alpha \in \mathbb{R} \).

\vspace*{-1em}
\subsubsection*{2. Stability at the equilibrium point}

The Jacobian matrix of the system is
\[
    J = \begin{bmatrix}
        \frac{\partial \dot{x}}{\partial x} & \frac{\partial \dot{x}}{\partial y} \\
        \frac{\partial \dot{y}}{\partial x} & \frac{\partial \dot{y}}{\partial y}
    \end{bmatrix}
    = \begin{bmatrix}
        \beta  & 0  \\
        \alpha & -1
    \end{bmatrix}
\]
which doesn't depend on the values of \( x \) and \( y \), thereby is same for all the equilibrium points.

The eigenvalues of the \( J \) can be found in this case by inspection as \( \lambda = \{ \beta, -1 \} \), since it is a lower triangular matrix.
For the system to be stable, the real part of the eigenvalues must be negative.
Therefore, the \underline{system is stable} at the equilibrium point for \( \boxed{ \beta < 0, \ \alpha \in \mathbb{R} } \).

\vspace*{-1em}
\subsubsection*{3. Instability at the equilibrium point}

Similary, we can see that with \( \boxed{\beta > 0, \ \alpha \in \mathbb{R}} \), the \underline{system is unstable} at the equilibrium point.
